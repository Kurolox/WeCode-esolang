% Full instructions available at:
% https://github.com/elauksap/focus-beamertheme

\documentclass{beamer}
\usetheme[numbering=none]{focus}
\usepackage[utf8x]{inputenc} %So spanish accents will work%
\usepackage{hyperref}
\usepackage{listings}
\usepackage{color}
\usepackage{verbatim}
\usepackage{inconsolata}
\usepackage{tikz-cd}
\usepackage{multicol}

\definecolor{codegray}{rgb}{0.5,0.5,0.5}
\definecolor{codecyan}{RGB}{255, 0, 0}
\definecolor{codeyellow}{RGB}{0, 85, 102}

\lstdefinestyle{mystyle}{
    commentstyle=\color{codeyellow},
    keywordstyle=\color{magenta},
    numberstyle=\color{codegray},
    stringstyle=\color{codecyan},
    basicstyle=\ttfamily\small,
    breakatwhitespace=false,         
    breaklines=true,                 
    captionpos=b,                    
    keepspaces=true,
    showspaces=false,                
    showstringspaces=false,
    showtabs=false,                  
    tabsize=2
}

\lstset{style=mystyle, escapeinside=||}
\title{Lenguajes Esotéricos}
\subtitle{De 0 a - [ - - - - - > + <] > - - . - . .}

\titlegraphic{\includegraphics[scale=0.13]{latex/static/fuck.png}}
\institute{WeCodeFest 2019}
\date{\today}

\begin{document}
\setbeamertemplate{caption}{\raggedright\insertcaption\par}
\begin{frame}
        \maketitle
\end{frame}
    % PARAGRAPH SPACING. -----------------------------------------------------------------
    \setlength{\parskip}{\baselineskip}%
    \setlength{\parindent}{0pt}%

\begin{frame}{Presentación}
\centering
{\Large Carlos Gómez (@Kurolox)\bigskip\pause

 Link a la presentación:}

\centering\textbf{\hyperlink{https://github.com/Kurolox/WeCode-esolang}{\Large https://github.com/Kurolox/WeCode-esolang}}
\end{frame}

\begin{frame}{Esolang 101}
    	\pause
        Un lenguaje esotérico (esolang) es un lenguaje de programación creado con el único propósito de experimentar con un concepto, o tener fines humorísticos.
\end{frame}
\begin{frame}{esolangs.org}
\textbf{\hyperlink{https://esolangs.org}{https://esolangs.org}} es una wikia dedicada a indexar y documentar lenguajes esotéricos.\pause

Ahora mismo, hay 1364 esolangs en esolangs.org.
\end{frame}

\begin{frame}{INTERCAL}

INTERCAL (Compiler Language With No Pronounceable Acronym) es el primer esolang del que se tiene registro, creado en 1972.

\textbf{\hyperlink{https://www.jdoodle.com/compile-intercal-online}{https://www.jdoodle.com/compile-intercal-online}} es un compilador online de INTERCAL.\pause

C también fue creado en 1972, por cierto.
\end{frame}

\begin{frame}[fragile]{INTERCAL: Hello World}
\begin{columns}
\begin{column}{0.65\textwidth}
\begin{lstlisting}
DO ,1 <- #13
PLEASE DO ,1 SUB #1 <- #238
DO ,1 SUB #2 <- #108
DO ,1 SUB #3 <- #112
DO ,1 SUB #4 <- #0
DO ,1 SUB #5 <- #64
DO ,1 SUB #6 <- #194
DO ,1 SUB #7 <- #48
PLEASE DO ,1 SUB #8 <- #22
DO ,1 SUB #9 <- #248
DO ,1 SUB #10 <- #168
DO ,1 SUB #11 <- #24
DO ,1 SUB #12 <- #16
DO ,1 SUB #13 <- #162
PLEASE READ OUT ,1
PLEASE GIVE UP
    \end{lstlisting}
\end{column}
\begin{column}{0.35\textwidth}  %%<--- here
\begin{itemize}
    \item Intencionalmente complejo de leer y escribir
    \item GIVE UP para finalizar
    \item PLEASE
\end{itemize}
\end{column}
\end{columns}
\end{frame}

\begin{frame}{INTERCAL: Please}
El programa ha de tener cierto grado de "cordialidad" (aproximadamente entre 1/3 y 1/5)\pause

Si el programa era "poco cordial" o "demasiado cordial", el compilador se negaba a funcionar.\pause

Esto \textbf{NO} estaba documentado en ninguna parte.
    
\end{frame}

\begin{frame}[fragile]{Brainfuck}
    Brainfuck es uno de los esolangs mas famosos, creado en 1992.\pause
    
    Este es un "Hello World" en brainfuck:
    \begin{lstlisting}
 ++++++++++[>+++++++>++++++++++>+++>+<<<<-]>++.>+.
 +++++++..+++.>++.<<+++++++++++++++.>.+++.---
 ---.--------.>+.>.
\end{lstlisting}

\end{frame}

\begin{frame}[fragile]{Brainfuck: How it works}

Imagína un array de celdas de memoria, y un puntero que apunta a una celda de memoria:
\begin{lstlisting}
    [0, 0, 0, 0, 0, 0, 0, 0, 0, 0, 0, 0, 0]
     ^
\end{lstlisting}

Los operadores de brainfuck son los siguientes:
\begin{multicols}{2}
\begin{itemize}
    \item + Suma 1 a la celda
    \item - Resta 1 a la celda
    \item \textless Mueve el ptr a la izqda
    \item \textgreater Mueve el ptr a la drcha
    \item . Imprime la celda
    \item , Guarda en la celda
    \item {[} Si ptr=0, salta tras el siguiente {]}
    \item {]} Salta al {[} anterior
    \end{itemize}
\end{multicols}
\end{frame}

\begin{frame}{Brainfuck: Proyectos notables}
\textbf{\hyperlink{https://fatiherikli.github.io/brainfuck-visualizer/}{https://fatiherikli.github.io/brainfuck-visualizer/}} es un compilador de brainfuck online en el que puedes visualizar paso a paso como se ejecutan las instrucciones.

\textbf{\hyperlink{https://github.com/matslina/awib}{https://github.com/matslina/awib}}
 es un compilador de brainfuck escrito en brainfuck
    
Tambien hay mas de 200 esolangs derivados de Brainfuck.\pause Por ejemplo, para los del taller de quantum computing de ayer...
\textbf{\hyperlink{https://esolangs.org/wiki/Quantum_brainfuck
}{https://esolangs.org/wiki/Quantum\_brainfuck}} 
\end{frame}

\begin{frame}[fragile]{LOLCODE y similares}
Hay una "categoría" de esolangs humorísticos en el que las instrucciones del lenguaje son extravagantes.

LOLCODE:
\begin{lstlisting}
HAI 1.3
I HAS A VAR ITZ 0
IM IN YR LP UPPIN YR VAR TIL BOTH SAEM VAR AN 5
  VISIBLE SUM OF VAR AN 1 
IM OUTTA YR LP
KTHXBYE
\end{lstlisting}
\end{frame}

\begin{frame}[fragile]{LOLCODE y similares}
TrumpScript
\begin{lstlisting}
make magic, not lies;
if Sanders is fact:
I'll tell you "Bush was never president"!
America is great.
\end{lstlisting}
    
\end{frame}

\begin{frame}[fragile]{LOLCODE y similares}
    Shakespeare:
    \begin{lstlisting}
    The Infamous Hello World Program.

Romeo, a young man with a remarkable patience.
Juliet, a likewise young woman of remarkable grace.
Ophelia, a remarkable woman much in dispute with Hamlet.
Hamlet, the flatterer of Andersen Insulting A/S.

        Act I: Hamlet's insults and flattery.
        Scene I: The insulting of Romeo.
        
[Enter Hamlet and Romeo]
Hamlet:
 You lying stupid fatherless big smelly half-witted coward!
 You are as stupid as the difference between a handsome rich brave
 hero and thyself! Speak your mind!

 You are as brave as the sum of your fat little stuffed misused dusty
 old rotten codpiece and a beautiful fair warm peaceful sunny summer's
 day. You are as healthy as the difference between the sum of the
 sweetest reddest rose and my father and yourself! Speak your mind!

 You are as cowardly as the sum of yourself and the difference
 between a big mighty proud kingdom and a horse. Speak your mind.

 Speak your mind!

[Exit Romeo]

                    Scene II: The praising of Juliet.

[Enter Juliet]

Hamlet:
 Thou art as sweet as the sum of the sum of Romeo and his horse and his
 black cat! Speak thy mind!

[Exit Juliet]

                    Scene III: The praising of Ophelia.

[Enter Ophelia]

Hamlet:
 Thou art as lovely as the product of a large rural town and my amazing
 bottomless embroidered purse. Speak thy mind!

 Thou art as loving as the product of the bluest clearest sweetest sky
 and the sum of a squirrel and a white horse. Thou art as beautiful as
 the difference between Juliet and thyself. Speak thy mind!

[Exeunt Ophelia and Hamlet]
        Act II: Behind Hamlet's back.
        Scene I: Romeo and Juliet's conversation.
[Enter Romeo and Juliet]
Romeo:
 Speak your mind. You are as worried as the sum of yourself and the
 difference between my small smooth hamster and my nose. Speak your
 mind!

Juliet:
 Speak YOUR mind! You are as bad as Hamlet! You are as small as the
 difference between the square of the difference between my little pony
 and your big hairy hound and the cube of your sorry little
 codpiece. Speak your mind!

[Exit Romeo]

                    Scene II: Juliet and Ophelia's conversation.

[Enter Ophelia]

Juliet:
 Thou art as good as the quotient between Romeo and the sum of a small
 furry animal and a leech. Speak your mind!

Ophelia:
 Thou art as disgusting as the quotient between Romeo and twice the
 difference between a mistletoe and an oozing infected blister! Speak
 your mind!

[Exeunt]
    \end{lstlisting}
\end{frame}

\begin{frame}{Malbolge: El lenguaje para no programar}

Malbolge es un esolang creado en 1998, y su nombre viene de \textit{Malebolge}, el noveno circulo del infierno de Dante.

Es un lenguaje diseñado explícitamente para que no pueda ser usado para programar.

El primer programa hecho en Malbolge apareció dos años tras la creación del lenguaje, y fue logrado a través de fuerza bruta.
    
\end{frame}

\begin{frame}[fragile]{Malbolge}
Hello world:
\begin{lstlisting}
 (=<`#9]~6ZY32Vx/4Rs+0No-&Jk)"Fh}|Bcy?`=*z]Kw%oG4UUS0/@-ejc(:'8dc
\end{lstlisting}

echo:
\begin{lstlisting}
 (=BA#9"=<;:3y7x54-21q/p-,+*)"!h%B0/.
~P<
<:(8&
66#"!~}|{zyxwvu
gJ%
\end{lstlisting}
\end{frame}

\begin{frame}{Malbolge: Funcionamiento}
\pause
    \begin{figure}
        \centering
        \includegraphics[width=\textwidth]{latex/static/noidea.png}
        \caption{Caption}
        \label{fig:my_label}
    \end{figure}
\end{frame}

\begin{frame}[fragile]{Befunge}

Befunge es un esolang el cual utiliza una estructura bidimensional y un stack, y utiliza flechas (\textless, \textgreater, v, \textasciicircum ) para dirigir el sentido del código.

Aquí hay un hola mundo:
\begin{lstlisting}
 >              v
v  ,,,,,"Hello"<
>48*,          v
v,,,,,,"World!"<
>25*,@
\end{lstlisting}
    
Podéis visualizar el funcionamiento de un programa de befunge en \textbf{\hyperlink{https://www.bedroomlan.org/tools/befunge-playground
}{https://www.bedroomlan.org/tools/befunge-playground}} 
\end{frame}
\begin{frame}{Un paso mas allá: Seed}
Seed es un esolang extremadamente simple. Solo tiene dos instrucciones:
\begin{enumerate}
    \item Una longitud (en bytes)
    \item Una "seed"
\end{enumerate}

Con esas dos instrucciones, Seed genera una string aleatoria de la longitud dada, la cual se pasa a un interprete de Befunge.

Compilador online: \textbf{\hyperlink{https://tio.run/#seed}{https://tio.run/#seed}} 
\end{frame}

    \begin{frame}{Code Golfing}
¿Quien de aqui sabe lo que es el code golfing? \pause

Code Golfing es una competición donde se busca resolver un problema especifico utilizando el menor numero de caracteres. 

Si os interesa el tema, \textbf{\hyperlink{https://codegolf.stackexchange.com/}{https://codegolf.stackexchange.com/}} 

Aquí tenéis un ejemplo:

Haz un programa con el menor numero de caracteres que imprima "Hello, World!".
\end{frame}
\begin{frame}[fragile]{Java}
\begin{lstlisting}
interface A{static void main(String[]a){System.out.print("Hello, World!");}}
\end{lstlisting}
\end{frame}

\begin{frame}[fragile]{Python}
\begin{lstlisting}
print"Hello, World!"
\end{lstlisting}
\end{frame}

\begin{frame}[fragile]{PHP}
\begin{lstlisting}
Hello, World!
\end{lstlisting}
\end{frame}

\begin{frame}[fragile]{Help, WarDoq!}
\begin{lstlisting}
W
\end{lstlisting}
\end{frame}

\begin{frame}{Stuck}

% Aqui empieza el Hello World de whitespace
   			 		   		  	 
    
		    	  	   
	
     		  	 	
	
     		 		  
 
 	
  	
     		 				
	
     	     
	
     	 	 			
	
     		 				
	
     			  	 
	
     		 		  
	
     		  	  
	
     	    	
	
     	 	 
	
   



% Aqui termina el hello world de whitespace
\end{frame}

\begin{frame}{Cambiando de tema, Whitespace}
Whitespace es un esolang en el que solo hay tres caracteres aceptados y el resto es ignorado:
\begin{itemize}
    \item Espacios
    \item Tabulador
    \item Salto de linea
\end{itemize}\pause

Os he metido el hola mundo en la diapositiva anterior.
\end{frame}

\begin{frame}{JavaScript (de verdad)}
Cuantos de aquí sois de frontend? \pause

Vamos a jugar a un juego rápido para ver que tal lleváis JavaScript.
    
\end{frame}

\begin{frame}[fragile]{Ronda 1}

A que evalúa []+[]?\pause

\begin{lstlisting}
 ""
\end{lstlisting}

\end{frame}

\begin{frame}[fragile]{Ronda 2}
Vamos con una mas difícil...
\begin{lstlisting}[basicstyle=\tiny]
 [][(![]+[])[+[]]+([![]]+[][[]])[+!+[]+[+[]]]+(![]+[])[!+[]+!+[]]+(!![]+[])[+[]]+(!![]+[])[!+[]+!+[]+!+[]]+(!![]+[])[+!+[]]][([][(![]+[])[+[]]+([![]]+[][[]])[+!+[]+[+[]]]+(![]+[])[!+[]+!+[]]+(!![]+[])[+[]]+(!![]+[])[!+[]+!+[]+!+[]]+(!![]+[])[+!+[]]]+[])[!+[]+!+[]+!+[]]+(!![]+[][(![]+[])[+[]]+([![]]+[][[]])[+!+[]+[+[]]]+(![]+[])[!+[]+!+[]]+(!![]+[])[+[]]+(!![]+[])[!+[]+!+[]+!+[]]+(!![]+[])[+!+[]]])[+!+[]+[+[]]]+([][[]]+[])[+!+[]]+(![]+[])[!+[]+!+[]+!+[]]+(!![]+[])[+[]]+(!![]+[])[+!+[]]+([][[]]+[])[+[]]+([][(![]+[])[+[]]+([![]]+[][[]])[+!+[]+[+[]]]+(![]+[])[!+[]+!+[]]+(!![]+[])[+[]]+(!![]+[])[!+[]+!+[]+!+[]]+(!![]+[])[+!+[]]]+[])[!+[]+!+[]+!+[]]+(!![]+[])[+[]]+(!![]+[][(![]+[])[+[]]+([![]]+[][[]])[+!+[]+[+[]]]+(![]+[])[!+[]+!+[]]+(!![]+[])[+[]]+(!![]+[])[!+[]+!+[]+!+[]]+(!![]+[])[+!+[]]])[+!+[]+[+[]]]+(!![]+[])[+!+[]]]((![]+[])[+!+[]]+(![]+[])[!+[]+!+[]]+(!![]+[])[!+[]+!+[]+!+[]]+(!![]+[])[+!+[]]+(!![]+[])[+[]]+(![]+[][(![]+[])[+[]]+([![]]+[][[]])[+!+[]+[+[]]]+(![]+[])[!+[]+!+[]]+(!![]+[])[+[]]+(!![]+[])[!+[]+!+[]+!+[]]+(!![]+[])[+!+[]]])[!+[]+!+[]+[+[]]]+([]+[])[(![]+[])[+[]]+(!![]+[][(![]+[])[+[]]+([![]]+[][[]])[+!+[]+[+[]]]+(![]+[])[!+[]+!+[]]+(!![]+[])[+[]]+(!![]+[])[!+[]+!+[]+!+[]]+(!![]+[])[+!+[]]])[+!+[]+[+[]]]+([][[]]+[])[+!+[]]+(!![]+[])[+[]]+([][(![]+[])[+[]]+([![]]+[][[]])[+!+[]+[+[]]]+(![]+[])[!+[]+!+[]]+(!![]+[])[+[]]+(!![]+[])[!+[]+!+[]+!+[]]+(!![]+[])[+!+[]]]+[])[!+[]+!+[]+!+[]]+(!![]+[][(![]+[])[+[]]+([![]]+[][[]])[+!+[]+[+[]]]+(![]+[])[!+[]+!+[]]+(!![]+[])[+[]]+(!![]+[])[!+[]+!+[]+!+[]]+(!![]+[])[+!+[]]])[+!+[]+[+[]]]+(![]+[])[!+[]+!+[]]+(!![]+[][(![]+[])[+[]]+([![]]+[][[]])[+!+[]+[+[]]]+(![]+[])[!+[]+!+[]]+(!![]+[])[+[]]+(!![]+[])[!+[]+!+[]+!+[]]+(!![]+[])[+!+[]]])
\end{lstlisting}
    
\end{frame}

\begin{frame}{JSFuck}
\begin{enumerate}
    \item Utiliza JavaScript \pause
    \item Limita tus entradas a las siguientes: {[} {]} + {(} {)} ! \pause
    \item ??? \pause
    \item JSFuck \pause
\end{enumerate}
\centering
\textbf{\hyperlink{http://www.jsfuck.com}{http://www.jsfuck.com}} 
\end{frame}

\begin{frame}{Y mucho mas...}
\begin{multicols}{2}
\begin{itemize}
    \item \hyperlink{https://esolangs.org/wiki/RTFM}{RTFM}
    \item \hyperlink{https://esolangs.org/wiki/Wierd}{Wierd}
    \item \hyperlink{https://esolangs.org/wiki/Fugue}{Fugue}
    \item \hyperlink{https://esolangs.org/wiki/DNA-Sharp}{DNA#}
    \item \hyperlink{https://esolangs.org/wiki/Pi}{Pi}
    \item \hyperlink{https://esolangs.org/wiki/Pit}{Pit}
\end{itemize}
\end{multicols}
    
\end{frame}
\end{document}